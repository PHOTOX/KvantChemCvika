\documentclass[a4paper,12pt]{article}
\usepackage[utf8]{inputenc}
\usepackage[T1]{fontenc} %řeší chyby se zalamovanim radku
\usepackage[czech]{babel}
\usepackage{graphicx}
\usepackage{amsfonts}
\usepackage{amssymb} %matematické fonty
\usepackage{epstopdf} %vkládání eps obrázků do pdf souboru
\usepackage{color}
\usepackage{picinpar}
\usepackage{booktabs} %lepsi vzhled tabulek
\usepackage[top=2cm,bottom=2cm,right=2.cm,left=2.0cm]{geometry}  %%%okraje
\usepackage[bf]{caption}   %tucne "Figure" a "table'
\captionsetup{width=0.9\textwidth} %úprava šířky popisků
\captionsetup{textfont=it, font=small} %úprava fontů u obrázků, text bude menší a italikou

\graphicspath{{obrazky/}{../obrazky/}} %adresar pro obrazky

%kontrola vdov a sirotku
\usepackage[defaultlines=4,all]{nowidow}

\usepackage[pdftitle={High-Performance computing},
pdfauthor={J. Chalabala, D. Hollas, P. Slavíček},
bookmarks=true,
colorlinks=true,
breaklinks=true,
urlcolor=red,
citecolor=blue,
linkcolor=blue,
unicode=true,
pdfstartview=FitV]{hyperref}

%\usepackage[super,sort&compress,comma]{natbib}
%%%%%%%%%%%%%%%%%%%%%%%%%%%%%%%%%%%%%%%%%%%%%%%%%%%%%%%%%%%%%%%%%%%%%%%%%%%%%%%%%%%

% Nazev budeme muset ještě vymyslet
\title{Moderní vědecko-technické výpočty}
\author{J. Chalabala, D. Hollas, P. Slavíček}

\begin{document}
\pagestyle{empty}
\maketitle

\tableofcontents

\pagestyle{plain}
\pagenumbering{arabic}

\clearpage
\section{Pro koho je text určen?}


\clearpage
\section{Trochu z počítačové historie}
ENIAC, salove pocitace, 
valecne vypocty: projekt Manhattan, Enigma-BOMB 

\section{Současnost - Superpočítače a počítačové klastry}

\section{Výpočetní možnosti v EU a ČR}

Možnost zažádat si o grant - Julich atp. zeptat se Pepy Melcra

Klastry na jednotlivých školách, IT4Inovation, metacentrum

\subsection{Výpočetní možnosti VŠCHT}

\end{document}
